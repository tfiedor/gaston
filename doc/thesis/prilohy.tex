\chapter{Contents of CD}

In main directory of the cd, there is a \texttt{cmake} source for compilation of
the application.

\paragraph{Directory \texttt{/doc/thesis}}

Contains \LaTeX sources with Makefile for compilation of thesis.

\paragraph{Directory \texttt{/examples/formulae}}

Contains examples written in \textsc{MONA} syntax for deciding. Ranging from
simple examples used for testing functionality to more complex ones used for
evaluation of created code.

\paragraph{Directory \texttt{/include/vata}}

Contains headers needed to be included during compilation of program for part of
the functions that are used from library \texttt{VATA}.

\paragraph{Directory \texttt{/src/app/DecisionProcedure}}

Contains main part of the application sources that does the conversion of given
representation of formulae into automaton and afterwards the procedure of
deciding validity or satisfiability.

\paragraph{Directory \texttt{/src/app/Frontend}}

Contains part of the sources that does the parsing of the input formulae
specification into inner representation in program

\paragraph{Directory \texttt{/src/libs}}

Contains external libraries that are used in application. Namely these are
several libraries needed for \textsc{MONA} part and library for manipulating
with NTA \textsc{VATA}.

\chapter{WS$k$S specification syntax}\label{syntax}

The following grammar below describes supported subset of \textsc{MONA} syntax
for specification of verified formulae. It uses the classical BNF-like notation.
For full syntax for \textsc{MONA} program consult the tool manual \cite{mona}.

\subsubsection{Specification}
\begin{verbatim}
program ::= (header;)? (declaration;)+
header ::=  ws1s | ws2s
\end{verbatim}

\subsubsection{Declarations}
% TODO: Not complete
\begin{verbatim}
declaration ::= formula
			 |  var0 (varname)+
             |  var1 (varname)+
             |  var2 (varname)+
             |  'pred' varname (params)? = formula
             |  'macro' varname (params)? = formula
\end{verbatim}

\subsubsection{Formulae}
\begin{verbatim}
formula ::= 'true' | 'false' | (formula)
         |  zero-order-var
         | ~formula
         | formula | formula
         | formula & formula
         | formula => formula
         | formula <=> formula
         | first-order-term = first-order-term 
         | first-order-term ~= first-order-term 
         | first-order-term < first-order-term 
         | first-order-term > first-order-term
         | first-order-term <= first-order-term 
         | first-order-term >= first-order-term
         | second-order-term = second-order-term
         | second-order-term ~= second-order-term
         | second-order-term 'sub' second-order-term
         | first-order-term 'in' second-order-term
         | ex1 (varname)+ : formula
         | all1 (varname)+ : formula
         | ex2 (varname)+ : formula
         | all2 (varname)+ : formula
\end{verbatim}

\subsubsection{First-order terms in WS1S}
\begin{verbatim}
first-order-term ::= varname | (first-order-term)
                  |  int
                  | first-order-term + int
                  | first-order-term - int
\end{verbatim}

\subsubsection{Second-order terms in WS1S}
\begin{verbatim}
second-order-term ::= varname | (second-order-term)
                   |  second-order-term union second-order-term
                   |  second-order-term inter second-order-term
                   |  second-order-term \ second-order-term
                   |  second-order-term + int
                   |  second-order-term - int
\end{verbatim}

\chapter{Usage}
The usage of the decision procedure tool is:
\begin{center}
 \texttt{dip [options] <filename>}
\end{center}
\texttt{<filename>} is relative or absolute path to specification of WS$k$S
formula as defined by syntax described in Appendix \ref{syntax}. The options
that can be further set are following:
\begin{itemize}
  \item[\texttt{-t}],\texttt{--time}\,--\,prints elapsed time for decision
  procedure and further information about timing of procedure.
  \item[\texttt{-d}],\texttt{--dump-all}\,--\,dumps information about AST
  representation of given formula, symbol table, created automaton and etc.
  \item[\texttt{-q}],\texttt{--quiet}\,--\,suppress printing of information
  about decision process.
  \item[]\texttt{--reorder-bdd}\,--\,by default variables are reorder according
  to the prefix of the given formula. This can be suppressed by adding parameter
  \texttt{no} or random reordering can be done by option \texttt{random}.
\end{itemize}

\chapter{List of Atomic Formulae}

We list in this Appendix automata corresponding to the atomic formulae of WS$k$S
logic used in decision procedure. These automata are further used for
construction of initial base automaton as described in Section \ref{wsks}. This
appendinx is structured into two parts: first describes basic set of atomic
formulae defined for restricted syntax (see \ref{restricted}) and the other
lists automata for the extension of syntax we are supporting to optimize size of
used automata.

Note that all shown automata are non-deterministic.

\section{Atomic formulae for restricted syntax}

\begin{figure}[h!]
 \begin{center}
  \includegraphics{fig/atomic-equal-succ}
 \end{center}
 \caption{Automaton corresponding to atomic formulae $X = Y1$, i.e. $Y$ is
 successor of $X$.}
\end{figure}

\begin{figure}[h!]
 \begin{center}
  \includegraphics{fig/atomic-equals-zero}
 \end{center}
 \caption{Automaton corresponding to atomic formulae $X = \epsilon$.}
\end{figure}

\begin{figure}[h!]
 \begin{center}
  \includegraphics{fig/atomic-equal-terms}
 \end{center}
 \caption{Automaton corresponding to atomic formulae $T_1 = T_2$, where $T_1$
 and $T_2$ are two second-order variables.}
\end{figure}

\begin{figure}[h!]
 \begin{center}
  \includegraphics{fig/atomic-singleton}
 \end{center}
 \caption{Automaton corresponding to atomic formulae $\mathit{Singleton}(P)$,
 i.e. that $P$ is set containing exactly one element.}
\end{figure}

\begin{figure}[h!]
 \begin{center}
  \includegraphics{fig/atomic-subset}
 \end{center}
 \caption{Automaton corresponding to atomic formulae $X \subseteq Y$}
\end{figure}

\section{Extending restricted syntax}

\begin{figure}[h!]
 \begin{center}
  \includegraphics{fig/atomic-x-in-X}
 \end{center}
 \caption{Automaton corresponding to atomic formulae $x \in X$}
\end{figure}

\begin{figure}[h!]
 \begin{center}
  \includegraphics{fig/atomic-x-lesseq-y}
 \end{center}
 \caption{Automaton corresponding to atomic formulae $x \in X$}
\end{figure}


%\chapter{Konfigra�n� soubor}
%\chapter{RelaxNG Sch�ma konfigura�n�ho soboru}
%\chapter{Plakat}

