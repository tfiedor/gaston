%%%%%%%%%%%%%%%%%%%%%%%%%%%%%%%%%%%%%%%%%%%%%%%%%%%%%%%%%%%%%%%%%
% Contents: Things you need to know
% $Id: things.tex 536 2015-06-26 06:41:33Z oetiker $
%%%%%%%%%%%%%%%%%%%%%%%%%%%%%%%%%%%%%%%%%%%%%%%%%%%%%%%%%%%%%%%%%
 
\chapter{Deciding WS1S}
\begin{intro}
In 1960~\cite{buchi} B\"{u}chi proved that there exists a one-to-one
correspondence between finite automata and formulae of WS1S logic.
This gave grounds for most of the popular decision procedures that 
are based on the construction of automata corresponding to the formulae.
Such automaton then accepts all of the models of the formulae.
\end{intro}

WS1S is Weak monadic logic with one successors:
quantify over:
\begin{itemize}
  \item \textbf{second-order}\,---\,can quantify over relations,
  \item \textbf{monadic}\,---\,unary relations (i.e. sets),
  \item \textbf{weak}\,---\,over finite sets,
  \item \textbf{one successor}\,---\,useful for describing linked structures.
\end{itemize}

Language-wise we can see the correspondence of WS1S logic with the
class of regular languages closed under the concatenations with 
strings of zero symbols (see \ref{sec:semantics}). WS1S is interpret
over infinite sets, but its quantification is restricted to finite
sets only\footnote{Note that there exists a variation of WS1S that is
interpreted over finite strings, which is often called M2L-str. 
Contrary to WS1S there exists a one to one correspondence with 
regular languages}.

\section{Automata-logic based connection}

\section{Symbolic decision procedure for WS1S}


%

% Local Variables:
% TeX-master: "lshort2e"
% mode: latex
% mode: flyspell
% End:
