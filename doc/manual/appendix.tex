\appendix
\chapter{Installing Gaston}
\begin{intro}

\end{intro}

\section{Installing dependencies}

Before installing the mighty Gaston you need the following friends.

\begin{itemize}
	\item[] git (>= 1.6.0.0)
	\item[] cmake (>= 2.8.2)
	\item[] gcc (>= 4.8.0)
	\item[] flex (>= 2.5.35)
	\item[] bison (>= 2.7.1)
	\item[] python (>= 2.0)
\end{itemize}
\tsf{i'm sure there are actually more of them}

\section{Setup and configure}

In order to compile and run \gaston first clone the source repository:

\begin{lstlisting}[language=bash]
 $ git clone https://github.com/tfiedor/Gaston.git
\end{lstlisting}

Go to source folder and run
\begin{lstlisting}[language=bash]
 $ make release
\end{lstlisting}
to run the Release version of \gaston.

In order to validate the installation and correctness of the tool
run the set of regressive tests:

\begin{lstlisting}[language=bash]
 $ python testbench.py
\end{lstlisting}

\chapter{Syntax of input formulae}

This chapter provides the supported syntax of the Gaston. For full
syntax of MONA formulae syntax conform the official MONA manual~\cite{mona:manual}.

\begin{verbatim}
program ::= (header;)? (declaration;)+
header ::=  ws1s } ws2s

declaration ::= formula
             |  var0 (varname)+
             |  var1 (varname)+
             |  var2 (varname)+
             |  'pred' varname (params)? = formula
             |  'macro' varname (params)? = formula

formula ::= 'true' | 'false' | (formula)
         |  zero-order-var
         | ~formula
         | formula | formula
         | formula & formula
         | formula => formula
         | formula <=> formula
         | first-order-term = first-order-term 
         | first-order-term ~= first-order-term 
         | first-order-term < first-order-term 
         | first-order-term > first-order-term
         | first-order-term <= first-order-term 
         | first-order-term >= first-order-term
         | second-order-term = second-order-term
         | second-order-term = { (int)+ }
         | second-order-term ~= second-order-term
         | second-order-term 'sub' second-order-term
         | first-order-term 'in' second-order-term
         | ex1 (varname)+ : formula
         | all1 (varname)+ : formula
         | ex2 (varname)+ : formula
         | all2 (varname)+ : formula

first-order-term ::= varname | (first-order-term)
                  |  int
                  | first-order-term + int

second-order-term ::= varname | (second-order-term)
                   |  second-order-term + int
\end{verbatim}
\tsf{better representation of the syntax}

\chapter{Command line interface}

\begin{lstlisting}[language=bash]
	Usage: gaston [options] <filename>
\end{lstlisting}

\begin{itemize}
	\item[\texttt{-t}, \texttt{--time}] Prints the elapsed time for each
		phase of the decision procedure
	\item[\texttt{-d}, \texttt{--dump-all}] Prints additional statistics
		about each phases (symbol table, etc.)
	\item[\texttt{-ga}, \texttt{--print-aut}] Outputs the resulting 
		automaton in graphviz format
	\item[\texttt{--no-automaton}] Does not output the resulting
		automaton
	\item[\texttt{--test=OPT}] Tests either satisfiability (\texttt{sat}),
		validity (\texttt{val}) or unsatisfiability (\texttt{unsat}) only
	\item[\texttt{--walk-aut}] Will walk the input formula and try to
		convert each subformula to automaton and prints statistics
	\item[\texttt{-e},\texttt{--expand-tagged}] Expands the automata
		with specific tags that are specified in the first line of
		the input formulae
	\item[\texttt{-q},\texttt{--quite}] Will not print any progress,
		only minimalistic informations
	\item[\texttt{-oX}] Sets optimization level (deprecated)
\end{itemize}

\chapter{Benchmarks}

This appendix sums the benchmarks that are used for evaluation of the tools for WS1S logic.
Every benchmark is briefly described, and quantified with several measures\,---\,number
of occurring variables (both bound and free), number of atomic formulae, and whether it is
valid, satisfiable or unsatisfiable. Note that we do not present some of the other measures
that are used in our experiments, like number of nodes in tree representation of formula, 
number of fixpoint computations and others, as they can vary from the used optimizations
and preprocessing steps.

\section{STRAND: Structure and Data}

This benchmark is obtained from the work of~\cite{strand1, strand2}. These formulae encode
various invariants of the cycles of several chosen algorithms (e.g. insert, delete) over 
various structures (e.g. linked lists). This benchmark is divided into two parts
\strandbenchone and \strandbenchtwo. The first one is the initial attempt to use the
WS1S for deciding the structural invariants, while the latter is the optimization of the 
initial one by reducing the size of the formulae and introducing some of the advanced 
constructs (e.g. predicates) that enabled faster decision procedure.

\begin{table}
  \caption{}\label{tab:bench-strand}
  \begin{tabular}{l l r r r}
  \hline
  \textbf{Id} & \textbf{Name} & \textbf{Variables} & \textbf{Atoms} & \textbf{Answer} \\
  \hline
  \hline
  \multicolumn{5}{*}{\strandbenchone}\\
  \hline
  \input{data\bench-strand-old}
  \hline
  \multicolumn{5}{*}{\strandbenchtwo}\\
  \hline
  \input{data\bench-strand-new}
  \hline
  \end{tabular}
\end{table}


