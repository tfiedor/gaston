%%%%%%%%%%%%%%%%%%%%%%%%%%%%%%%%%%%%%%%%%%%%%%%%%%%%%%%%%%%%%%%%%
% Contents: Things you need to know
% $Id: things.tex 536 2015-06-26 06:41:33Z oetiker $
%%%%%%%%%%%%%%%%%%%%%%%%%%%%%%%%%%%%%%%%%%%%%%%%%%%%%%%%%%%%%%%%%
 
\chapter{Introduction}
\begin{intro}
Long ago, there was a brave soldier that lived behind the bars and
led a fun splashy life. Then one day water have risen and thanks to
the flood, the brave soldier escaped from his capture and swam all
the way to the Dresden. Here he was caught and during his return
back to custody he sadly succumbed to the diseases and died before
seeing his home again.

This manual is dedicated to Gaston, the brave sea lion from the Prague 
Zoo, that gave name to our tool.
\end{intro}

\section{Introduction}

Logics are widely used as a great tool for describing invariants,
models and various properties used in verification tools.
WS1S is weak monadic second order logic with one successor (or WS1S).
Its applications are wide, mostly because of the successful 
implementation of its classical decision procedure in the \mona tool.
This tool has been the fastest, however as the theoretical 
complexity of deciding WS1S is in \nonelementary~\cite{ws1s:nonelementary} class,
there are still some limitations and some classes of formulae,
where \mona runs out of memory. 

However, we try to push the usability border even further and try to
develop a novel decision procedure. 

\section{Related Work and Application of WS1S logic}

\gaston was built on the initial work conducted in~\cite{dwina}, where we first 
tried to address the issue of repeated automata determinization, as the 
quantifier alternation is the main source of the \nonelementary complexity
of deciding the WS1S logic. In this work we exploited the recent advancements
in testing of language inclusion and universality based on the so-called antichains
~\cite{antichains, wulf:antichains}. The proposed approach suggested not to construct the automaton
explicitely, but only use the final states represented symbolically using anti-chains. 
Such antichains are nested, as the determinization is performed for every quantifier
alternation, with one nesting for each alternation. 

However, this work suffered from several issues, that we tried to address in \gaston
by generalizing the structure of the terms and introduce the laziness to the procedure.
The approach was restricted to formulae in Prenex Normal Form, and moreover could
suffer from state explosion at the base level of the state space, as no minimization took
place during the constructions (which \mona exploits).

The foremost implementation of WS1S logic, that enabled its application in wide areas
of computer science, is the \mona tool~\cite{mona}. Its efficiency stems from wide
number of optimizations both higher-level (such as automata minimization, encoding of the
first-order variables used in models, or the use of BDDs to encode the transition relation
\footnote{note that this is the most important optimizations, as no serious implementation
of decision procedure could be based on naive table representation of the transition relation}
as well as dozens of lower-level ones (e.g. optimizations of hash tables)~\cite{mona:secrets,
mona:relativization}.

Apart from \mona, there are some other related approaches that are based on the classical
automata-based approach such as \jmosel \cite{jmosel} for a related logic \msl, which
introduces some other orthogonal optimizations such as second-order value numbering~\cite{jmosel:dag},
that allows \jmosel to outperform other tools.

Recently, apart from our approaches, a couple of interesting logic-based approaches 
for WS1S logic appeared. Ganzow and Kaiser~\cite{kaiser} developed a new decision procedure
for the weak monadic second-order slogic on inductive structures, which is more general
than WS$k$S, completely avoiding usage of automata. Instead it is based on Shelah's 
composition method. Traytel~\cite{traytel}, on the other hand, build his approach on
classical decision procedure, however in the framework of coalgebras. His work focuses
on testing equivalence of a pair of formulae, which is performed by finding a bisimulation
between derivatives of the formulae. 

Further we list some of  the most recent applications of WS1S in practice. For full list
of either WS1S application or the direct usage of implementation of the \mona tool conform
the \mona manual~\cite{mona:manual}.

\tsf{Add several tools that are not in mona manual, like regsys, bow-yaw, etc.}

\section{Related Tools}

This section describes some other tools and prototypes that can be 
used for deciding of WS1S formulae. Further we discuss their 
limitations and compare their main usage.

Table~\ref{tab:tools-comparison} shows comparisons between the most recent tools
that implement decision procedure for WS1S logic or can be considered as 
state-of-the art. Out of the five chosen tools we can see the limitations of all
approaches, with only \gaston and \mona being capable of deciding unground formulae.

The \jmosel tool is tuned for the \msl logic \footnote{note that the decision
procedures for \msl and WS1S is almost identical with exception of the phase
of saturation}, which means it can be considered
incomparable to other tools. However, its approach introduces interesting 
orthogonal optimizations and is widely used in practice.

The tools of~\cite{kaiser} and \cite{traytel} are novel and prototype tools, that
do not fully support the wide syntax available to \mona, as well as extensions and
other capabilities.

\begin{table}
\footnotesize
  \begin{tabular}{l|lll|ll|l|l}
\hline
\multicolumn{1}{c}{\multirow{2}{*}{Tool}} & \multicolumn{3}{|l|}{Supported logics} & \multicolumn{2}{l|}{Syntax} & \multirow{2}{*}{Models} & \multirow{2}{*}{Last update} \\
\multicolumn{1}{c}{} & WS1S & WS2S & \msl & Atomic & Predicates & & \\
\hline
 \gaston & \checkmark & $\times$ & \checkmark & Full & \checkmark & \checkmark & 07/05/2016\\
 \dwina & \checkmark & $\times$ & $\times$ & Partial & \checkmark & $\times$ & 05/02/2015\\
 \mona & \checkmark & \checkmark & \checkmark & Full & \checkmark & \checkmark & ??/??/????\\
 \jmosel & ?? & ?? & ?? & ?? & ?? & ?? & ??/??/????\\
 Traytel & ?? & ?? & ?? & ?? & ?? & ?? & ??/??/????\\
 Kaiser & ?? & ?? & ?? & ?? & ?? & ?? & ??/??/????\\
 \hline
\end{tabular}
 \caption{Comparison of tools capable of deciding WS1S logic, their support for
 other kinds of logics, different syntactic expressions, ability to generate
 (counter)examples and liveness of the development.}\label{tab:tools-comparison}
\end{table}

%

% Local Variables:
% TeX-master: "lshort2e"
% mode: latex
% mode: flyspell
% End:
